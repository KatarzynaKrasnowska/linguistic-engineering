\documentclass[12pt]{beamer}

\usepackage[utf8]{inputenc}
\usepackage{latexsym}
\usepackage{xcolor}
\usepackage{iwona}
\usepackage[round]{natbib}
\usepackage{fixltx2e} %for subscripts
\usepackage{amsmath}
\usepackage{amssymb}
\usepackage{bbm}
\usepackage{polski}

\mode<handout>{
    \usepackage{pgfpages}
    \pgfpagesuselayout{4 on 1}[a4paper,
    border shrink=5mm,landscape]
    \setbeamercolor{background canvas}{bg=black!3}
}
\definecolor{wlasny}{rgb}{.125,.5,.15}
\usecolortheme[named=wlasny]{structure}

\mode<beamer>{%
  \usetheme{IPI}
}

\AtBeginSection[]
{
   \begin{frame}
       \frametitle{Outline}
       \tableofcontents[currentsection]
   \end{frame}
}


\title{Extended tokenizer for Polish}
\author{Tomasz Bartosiak \\ Konrad Gołuchowski \\ Katarzyna Krasnowska}
\date{14 March 2014}

\begin{document}
%-------------------------------------------------------------------------------
\begin{frame}
    \titlepage
\end{frame}
%-------------------------------------------------------------------------------
\begin{frame}{Stage 1: token splitting}
    \begin{itemize}
        \item Text is split on spaces.
        \item Additionally, leading and trailing punctuation marks are separated:
            \begin{itemize}
                \item exception: dot preceded by non-punctuation,
                \item exception from exception: three consecutive dots.
            \end{itemize} 
    \end{itemize}
\end{frame}
%-------------------------------------------------------------------------------
\begin{frame}{Stage 2: token tagging (and further splitting)}
    \begin{itemize}
        \item Cascade of tag filters.
        \item Regular expression-based, e.g.:
            \begin{itemize}
                \item \texttt{rom}, \texttt{ara}, \texttt{e-mail}, \texttt{www}
                \item dates in formats 14.03.2014, 14.03.2014.
            \end{itemize}
        \item More complicated, e.g.:
            \begin{itemize}
                \item abbreviations,
                \item \textit{I}/\textit{i} conjunction,
                \item hyphen-separated tokens.
            \end{itemize}
        \item Helper tags:
            \begin{itemize}
                \item \texttt{int} for arabic integers,
                \item \texttt{date} for dates as above,
                \item \texttt{m-i}, ..., \texttt{m-xii} for month names.
            \end{itemize}
    \end{itemize}
\end{frame}
%-------------------------------------------------------------------------------
\begin{frame}{Stage 2: abbreviations}
    \begin{itemize}
        \item A list of about 1300 abbreviations is used.
        \item Dot-ended abbreviations identical with some other word's inflected form:
            \begin{itemize}
                \item e.g., \textit{giełd.}, \textit{gwar.}, \textit{ul.}
                \item heuristic: only tag as \texttt{abbrev} in the middle of a sentence.
            \end{itemize}
        \item Other dot-ended abbreviations:
            \begin{itemize}
                \item e.g., \textit{dot.}, \textit{egip.}, \textit{popr.}
            \end{itemize}
        \item Mutli-part abbreviations:
            \begin{itemize}
                \item e.g., \textit{m.in.}, \textit{p.n.e.}
            \end{itemize}
        \item Abbreviations without the dot:
            \begin{itemize}
                \item e.g., \textit{mjr}, \textit{s-ka}, \textit{EUR}, \textit{MB}
                \item can be inflected: \textit{dra}, \textit{OSiR-u}.
            \end{itemize}
    \end{itemize}
\end{frame}
%-------------------------------------------------------------------------------
\begin{frame}{Stage 3: date parsing}
    \begin{itemize}
        \item Straightforward for \texttt{date}-tagged tokens.
        \item Look for specific token/tag sequences, e.g.:
            \begin{itemize}
                \item tag=\texttt{int} - tag=\texttt{rom} - tag=\texttt{int}
                \item tag=\texttt{int} - tag=\texttt{m-$\ast$} - tag=\texttt{int} - tok=\texttt{"r"} - tok=\texttt{"."}
            \end{itemize}
        \item Check day and month range.
        \item Merge tokens into one and assign them appropriate tag.
        \item Replace remaining \texttt{int} and \texttt{m-$\ast$} tags with \texttt{ara} and \texttt{word} respectively.
    \end{itemize}
\end{frame}
%-------------------------------------------------------------------------------
\end{document}
